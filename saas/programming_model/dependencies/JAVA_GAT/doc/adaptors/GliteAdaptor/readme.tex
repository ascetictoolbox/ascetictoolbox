\documentclass{scrreprt}
\usepackage[latin1]{inputenc}
\usepackage[english]{babel}
%\usepackage{listing}
\usepackage{hyperref}

\begin{document}
\title{Glite Adapter Readme}
\author{Thomas Zangerl \and Max Berger}
\maketitle
\tableofcontents
		
\chapter{VOMS-Proxy Creation}

\section{Frequent errors}

\subsection{"Unknown CA" error}\label{unkn_ca}

The proxy classes report an "Unknown CA" error (Could not get stream from secure socket).
Probably the VomsProxyManager is missing either your root certificate
or the root certificate of the server you are communicating with.
It is best, if you include all needed certificates in ~/.globus/certificates/.
(e.g. you can copy the /etc/grid-security/certificates directory from an UI machine
of the VO you are trying to work with to that location).

If this doesn't suffice, you should try to include a file called cog.properties in the ~/.globus/
directory. The content of this file could be something like this:

\begin{verbatim}
Java CoG Kit Configuration File
#Thu Apr 05 15:59:23 CEST 2007
usercert=~/.globus/usercert.pem
userkey=~/.globus/userkey.pem
proxy=/tmp/x509up_u<your user id>
cacert=/etc/grid-security/certificates/
ip=<your ip address>
\end{verbatim}

Also check the vomsHostDN preference value for typos/errors.

\subsection{"Error while setting CRLs"}

Try to create a cog.properties file in ~/.globus and set the cacert property to ~/.globus/certificates.
This will cause the VOMS Proxy classes not to look for CRLs in /etc/grid-security/certificates

\subsection{"pad block corrupted"}\label{pad_block}

Check whether you have given all necessary information (password, host-dn, location
of your user-certificate etc., see section \ref{min_voms_config}) as global preferences of the GAT-context.

If you have given all necessary information, check the password you have given, for typos.

\subsection{Could not create VOMS proxy! failed: null:}

See answer in section \ref{pad_block}, maybe you forgot to specify the vomsServerPort preference value.

\section{Preference keys for VOMS Proxy creation}

The \textbf{necessary} preference keys are: \\

\noindent
\begin{tabular}{|l|p{4cm}|p{4cm}|l|}
\hline
vomsHostDN & distinguished name of the VOMS host & /DC=at/DC=uorg /O=org/CN=somesite & compulsory \\
vomsServerURL & URL of the voms server, without protocol & skurut19.cesnet.cz  & compulsory \\
vomsServerPort & port on which to connect to the voms server & 7001 & compulsory\\
VirtualOrganisation & name of the virtual organisation for which the voms proxy is created & voce & compulsory \\
vomsLifetime & he desired proxy lifetime in seconds & 3600 & optional \\
\hline
\end{tabular}


Additionally you need a CertificateSecurityContext which points to your user certificate and your user key.
Add that CertificateSecurityContext to the GATContext.

With the compulsory preferences, the proxy is created with a standard lifetime of 12 hours. If you want to have a different
lifetime, add the optional vomsLifetime preference key.

Do something like

\begin{verbatim}
GATContext context = new GATContext();
CertificateSecurityContext secContext = 
		new CertificateSecurityContext(new URI(your_key), new URI(your_cert), cert_pwd);
Preferences globalPrefs = new Preferences();
globalPrefs.put("vomsServerURL", "voms.cnaf.infn.it");
...
context.addPreferences(globalPrefs);
context.addSecurityContext(secContext);
\end{verbatim}

\section{Minimum configuration to make VOMS-Proxy creation work}\label{min_voms_config}

\begin{itemize}
 \item A cog.properties file with lines as in section \ref{unkn_ca}.
 \item The following global preferences set in the gat context
 \begin{itemize}
  \item vomsHostDN
  \item vomsServerURL
  \item vomsServerPort
  \item VirtualOrganisation 
 \end{itemize}
\end{itemize}

\section{(Not) reusing the VOMS proxy}

If multiple job submissions to the same VO happen in a small time interval, it is not necessary to create
a new VOMS proxy for every submission. 
Hence, if the user sets the system property glite.reuseProxy to true, the system will check, whether a valid VOMS-Proxy
exists. If such a proxy is found, it is determined, whether the proxy lifetime exceeds the one specified in
the vomsLifetime property. If the vomsLifetime property is unspecified, it is checked wether the proxy is still
valid for more than 10 minutes.
Very frequently, all job submissions of an application will happen within the same VO. In case there is the
need to submit jobs to multiple VOs, it won't be possible to to reuse the old proxy due to the VO-specific
attribute certificates stored in the proxy file. 

\chapter{The gLite Resource Broker Adaptor} 

\section{Adaptor-specific system properties}

Indeed, the mechanisms provided by the GAT-API alone did not suffice to provide all the control we found 
desirable for the adaptor. Hence, a few proprietary properties where introduced. They are useful
in controlling adaptor behaviour but are by no means necessary if one just wants to use the adaptor.
Nonetheless, they are documented here.
 
If you want to use them, set them using System.setProperty(); for example write 
System.setProperty("glite.pollIntervalSecs", "15").
The following properties are supported:

\begin{itemize}
 \item glite.pollIntervalSecs - how often should the job lookup thread poll the WMS for job status updates and fire MetricEvents with status updates (value in seconds, default 3 seconds)
 \item glite.deleteJDL - if this is set to true, the JDL file used for job submission will be deleted when the job is done ("true"/"false", default is "false")
 \item glite.newProxy - if this is set to true, create a new proxy even if the lifetime of the old is still sufficient
\end{itemize}

\section{Supported SoftwareDescription attributes}

The minimum supported attributes from the software description seem to be derived from the features
that RSL (the globus job submission file format) provides. Hence, they are easy to translate to RSL 
properties.
However, the format used for glite job submission is JDL and attributes like count or hostCount are
hard to translate to JDL. Most of the attributes that \textbf{are} supported are not even achieved by the JDL
format itself, but by adding GLUE requirements.
Sadly, the JDL format does not provide much of the functionality covered by RSLs, hence many attributes
remain unsupported.
 
On the other hand, to enable working with the features that the JDL format provides additionally to
the RSL format, a new attribute was introduced.
 
Set glite.retryCount to some String or Integer in order to use the retry count function of glite.

\section{Setting arbitrary GLUE requirements}

If you would like to specify any GLUE-Requirements that are not covered by the standard set of 
SoftwareResourceDescription or HardwareResourceDescription keys, you may set glite.other either as
Software- or HardwareResourceDescription and add a *full* legal GLUE Requirment as entry, such as
for example \\ \verb+!( other.GlueCEUniqueID == "some_ce_of_your_choice")+.

\chapter{File Management}

Please note that the file management adapters are still to be
considered experimental. The supported attributes are subject to change!

Supported Schemes are:

\begin{tabular}{ll}
\bf Schema&\bf Supported\\\hline
srm&Yes\\
guid&Yes\\
lfn&No
\end{tabular}

Supported attributes are:

\begin{tabular}{lllp{6cm}}
\bf Preference&\bf Req.&\bf Default&\bf Description\\\hline
VirtualOrganisation&Yes&& The VO to use\\
LfcServer&No&from LDAP&Address of the LFC Server\\
LfcServerPort&No&5010&Port for the LFC Server
\end{tabular}

Supported operations are:

\begin{tabular}{lllp{6cm}}
\bf Operation&\bf SRM&\bf Guid&\bf Remarks\\\hline
copy&Yes&Yes&Supports upload from and download to local files only.\\
delete&Yes&Yes&\\
createNewFile&No&Yes&File must be created with {\tt guid:///}. The
actual guid can be retrieved with the function {\em .toGATURI()}.
\end{tabular}

\end{document} 
