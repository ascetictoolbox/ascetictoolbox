\chapter{Running Extrae on top of PnMPI}

\section{Introduction}

Most tools targeting MPI rely on the MPI Profiling Interface (PMPI), which allows tools to transparently intercept invocations to 
MPI routines and with that to establish wrappers around MPI calls to gather execution information. However, the usage of this interface 
is limited to a single tool. PnMPI eliminates the restriction of a single PMPI tool layer per execution. It can dynamically load 
and chain multiple PMPI tools into a single tool stack and then interject this complete stack between the target application and the 
library without changing the view for each individual tool. It enables the user to combine arbitrary MPI tools without having to reimplement 
them. When Extrae is operating through LD\_PRELOAD interposition it also supports to run on top of PnMPI.

\section{Instructions to run with PnMPI}

The Extrae's tracing libraries have to be processed with the \"patch\" tool that comes included with PnMPI. Just run this utility, passing as 
argument the tracing library that you want to load under the PnMPI environment and the output name for the patched library. 

\begin{verbatim}

$PNMPI_HOME/patch/patch libmpitrace.so libmpitrace-pnmpi.so

\end{verbatim}

At execution time the PNMPI\_CONF environment variable has to be defined, pointing to a file that specifies all the tools that will be loaded
with PnMPI. 

\begin{verbatim}

export PNMPI_CONF=$PNMPI_HOME/demo/.pnmpi-conf

\end{verbatim}

In this file we have to add the patched tracing library at the beginning of the list.

\begin{verbatim}

module libmpitrace-pnmpi
module another-tool
...

\end{verbatim}
